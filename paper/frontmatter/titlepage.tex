%Header
\pagestyle{fancy}
\thispagestyle{empty}
\rhead{ \textit{ }} 

% Update your Headers here
% \fancyhead[LO]{Running Title for Header}
% \fancyhead[RE]{Firstauthor and Secondauthor} % Firstauthor et al. if more than 2 - must use \documentclass[twoside]{article}

%% Title
\title{
    \textbf{\paperName}
    \thanks{Исходный код, конфигурации и скрипты обучения доступны: \href{https://github.com/namenick91/Convformer}{https://github.com/namenick91/Convformer}} 
}

\author{
  \studentName \\
  % Affiliation \\
  % Univ \\
  % City\\
  \texttt{email@email} \\
    \And
  \teacherName \\
  % Affiliation \\
  % Univ \\
  % City\\
  \texttt{email@email} \\
}

% \begin{titlepage}
\maketitle

\begin{abstract}
  Задача долгосрочного прогнозирования многомерных временных рядов остается 
  сложной: в реальных данных наблюдается ряд проблемных эффектов - среди них, 
  краткосрочные мотивы (скачки, ступени), 
  крайне долгосрочные и межрядовые зависимости, выраженная нестационарность 
  (сдвиги тренда/уровня, мультисезонность, гетероскедастичность).
  В данной работе мы предлагаем расширить слой эмбеддинга в модели 
  Informer~\cite{informer} компактным двухслойным сверточным блоком с целью 
  повышения эффективности извлечения локальных паттернов, внедрить модуль
  декомпозиции ряда из Autoformer~\cite{autoformer} для явного разделения 
  трендовых и сезонных компонент и заменить механизм ProbSparse на линейное внимание FAVOR+
  (Performer~\cite{performer}) для учёта глобальных зависимостей при низких 
  вычислительных затратах.
  Эксперименты на стандартных бенчмарках демонстрируют
  снижение ошибок на длинных горизонтах прогнозирования в ряде настроек.
  Абляционные исследования указывают на вклад каждого из модулей, 
  хотя эффект не универсален для всех горизонтов. 
  На рассмотренных конфигурациях наблюдается благоприятный компромисс между точностью 
  и затратами времени/памяти.
\end{abstract}

% keywords can be removed
\keywords{
  LSTF \and Long Sequence Time-series Forecasting \and Transformer \and 
  Informer \and Performer \and Autoformer
}
% \end{titlepage}
