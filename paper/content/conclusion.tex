\section{Заключение}

Предложенная комбинация ConvStem + FAVOR+ + Autoformer-подобная декомпозиция демонстрирует 
улучшение качества на длинных горизонтах при сопоставимых вычислительных затратах 
на рассматриваемых настройках. Абляционные исследования подтверждают вклад 
модулей на большинстве горизонтов: ConvStem - выделение локальных мотивов; FAVOR+ - 
масштабируемые дальние зависимости; декомпозиция - устойчивость к тренду/сезонности. 
На ETT наибольшие выигрыши достигаются при длительных горизонтах (336-720), тогда как 
на средних горизонтах Autoformer остается сопоставимым. 

При этом исследование имеет ряд ограничений: в качестве бенчмарка был рассмотрен один 
набор данных (ETT), 
cross-attention сохраняет квадратичную сложность, чувствительность MAPE/MSPE к масштабам признаков, 
отсутствие оценки неопределённости. Перспективными направлениями являются
линеаризация cross-attention, адаптивный выбор ранга $r$ в FAVOR+, расширение набора датасетов, 
включение калибровки и методов, устойчивых к дрейфу распределений.
